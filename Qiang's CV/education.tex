% Don't edit this until you are graduated from Zhejiang University.
\begin{rubric}{Education}
\entry*[2019.05 -- 2023.09]
	\textbf{Ph.D. Student, Ph.D. Candidate (2020.09), Ph.D. (2023.09) Computer Science}
	\par {College of Computer Science, Zhejiang University, China}
	\par Thesis title: 
	\emph{Research on Key Technologies of Virtualization for Linux-based Peripherals}.
	\par Thesis statement:
	\emph{Virtualization technology is required to rehost Linux-based IoT
	devices on virtual execution environment (VEE) to suppport dynamic analysis,
	which has two objectives. First, the VEE should be as close as possible to
	the physical Linux-based IoT device (fidelity); second, each virtual
	Linux-based IoT device should be well isolated (security). We then propose
	two new technologies, respectively, 1) model-guided kernel execution, which
	ensures the fidelity of the whole VEE by constructing high-fidelity virtual
	Linux-based peripherals; 2) dependency-aware message model, which maintains
	the security of the whole VEE by fuzzing virtual Linux-based peripherals.
	Through the above two novel methods, we finally realize a high-fidelity and
	high-security VEE to analyze and mine vulnerabilities for Linux-based IoT
	devices.}

	\par Advisors: Yajin Zhou (Zhejiang University), Mathias Payer (EPFL)
	\par Collaborators: Cen Zhang, Lin Ma, Flavio Toffalini
\entry*[2018.09 -- 2019.05]
	\textbf{Ph.D. Student, Computer Science}
	\par {College of Computer Science, Zhejiang University, China}
	\par Advisor: Yan Chen (Northwestern University; Zhejiang University)
\entry*[2014.09 -- 2018.06]
	\textbf{Bachelor, Electrical Engineering}
	\par {School of Electrical Engineering, Beijing Institute of Technology, China} 
	\par Thesis title: 
	\emph{Applying LSTM to the implicit continuous authentication of smart phones}.
	\par Thesis statement:
	\emph{Through implicit continuous authentication system based on the 
	smart phone motion sensor, it is possible to solve the problems of ease of use
	and security in user authentication. With the LSTM model and parameters tuning, 
	the final FAR reached 6.352\% and the FRR reached 6.232\%. This result shows that
	the implicit continuous authentication has considerable accuracy, providing
	support for the introduction of implicit continuous authentication into existing
	smartphones.}
	\par Advisor: Yan Chen (Northwestern University; Zhejiang University)
	\par Co-advisors: Limin Pan and Senlin Luo (Beijing Institute of Technology)
	\par Mentor: Tiantian Zhu (\st{Zhejiang University;} Zhejiang University of Technology)
\end{rubric}